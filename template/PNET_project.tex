%---------------------------------------------------------------------
% This file provides a skeleton UCL DIS CDT report.
% \pdfinclusioncopyfonts=1
% This command may be needed in order to get \ell in PDF plots to appear. Found in
% https://tex.stackexchange.com/questions/322010/pdflatex-glyph-undefined-symbols-disappear-from-included-pdf
%---------------------------------------------------------------------
% Specify where LaTeX style files can be found.
\newcommand*{\DISCDTLATEXPATH}{latex/}
% Use this variant if the files are in a central location, e.g. $HOME/texmf.
%---------------------------------------------------------------------

\documentclass[NOTE, disdraft=true, UKenglish]{\DISCDTLATEXPATH UCLCDTDISdoc}
% The language of the document must be set: usually UKenglish or USenglish.
% british and american also work!
% Commonly used options:
%  cdtdraft=true|false   This document is an UCL CDT DIS draft.
%  paper=a4|letter       Set paper size to A4 (default) or letter.

%--------------------------------------------------------------------- 
% Add you own definitions here (file dis-gp-defs.sty).
%\usepackage{dis-gp-defs}
%---------------------------------------------------------------------


%--------------------------------------------------------------------- 
% Files with references for use with biblatex.
% Note that biber gives an error if it finds empty bib files.
%\usepackage{biblatex} % uncomment if use addbibresource command
%\addbibresource{dis-gp.bib}
%--------------------------------------------------------------------- 

% Paths for figures - do not forget the / at the end of the directory name.
\graphicspath{{logos/}{figures/}}


%---------------------------------------------------------------------
% Generic document information
%---------------------------------------------------------------------
% Title, abstract and document 
%-------------------------------------------------------------------------
% This file contains the title, author and abstract.
% It also contains all relevant document numbers if needed.
%-------------------------------------------------------------------------

% Partner Logo
% put the name of the logo image file found in graphics path
\DISPartnerLogo{logos/UCL_purple.png}


% Title
\DISTitle{Improving radiosensitivity predictions of brain cells using a biologically informed neural network}

% Draft version:
% If given, adds draft version on front page, a 'DRAFT' box on top of each other page, 
% and line numbers.
% Comment or remove in final version.
\DISVersion{0}

% Abstract - % directly after { is important for correct indentation
\DISAbstract{%
  One of the current most effective treatments for brain tumor's is radiotherapy. However, this has the side-effect of damaging healthy cells with potentially severe consequences. Improvements in radiotherapy treatment such as proton therapy allow a more spatially localized treatment. However, the sensitivity of different cell types to radioactivity is not currently well modeled. We present a method for predicting the radio-sensitivity of a cell based on its genetic content using a biologically informed neural network. We compare the use of different data features for training this model, different network architecture and develop a procedure for transfer learning, where drug response data is incorporated into the model training. We compare the model performance using a cross-validation procedure ensuring unbiased performance estimates with faithful uncertainty estimates. We find ...
}


% Authors and list of contributors to the analysis
\usepackage{authblk}
\author[a]{Toby Dixon}
\author[a]{Dakshesh Kololgi}
\author[b]{Jason Ran}

\affil[a]{University College London}


% DIS reference code if ever used
\DISRefCode{UCLCDTDIS-2024-XX}

% Author and title for the PDF file
\hypersetup{pdftitle={UCL CDT DIS Document},pdfauthor={The UCL CDT DIS}}

%---------------------------------------------------------------------
% Content
%---------------------------------------------------------------------
\begin{document}

\maketitle

\tableofcontents

\clearpage


%---------------------------------------------------------------------
\newpage
%---------------------------------------------------------------------

\newpage
%---------------------------------------------------------------------
\section{Introduction}
\label{sec:introduction}
%\input{introduction}
%---------------------------------------------------------------------
\section{Data}
\label{sec:data}
{\color{red}Toby: I think what we should do here is describe the available datasets (where they come from, the size the features), and make some nice plots to show the distributuons in data etc.}
\\
For this work we use data from the CCLE (full name+ \ref{}) database. This consists of measurements of the radio-sensitivity of 511 cell lines along with their genetic sequence. The radio-sensitivity is characterized by the area under the dose response curve (AUC). This is relatively expensive measure requiring irradiation of cells with known doses of radiation. This limits the size of the available dataset. 
\\ \indent However, advances in genetic sequencing \ref{} mean that for each cell line there is a large amount of genetic information available. Thus the dataset available has a large number of features but very few data-points. This situation can easily lead to overfitting \ref{}. We focus on 3 genetic features. Gene-expression, copy number violation, and mutations.
\section{Methods}
\label{sec:method}
For this work we employ a biologically informed neural network to predict the radio-sensitivity. This is based on our adaptation of P-NET \cite{}. A typical dense neural network has millions of connections, and thus weights to tune. Thus can easily lead to over-fitting where the model predicts well on the training dataset but not on unseen data. A solution to this is to encode some prior knowledge into the models architechture. In this way we create a sparse network where the connections represent known biological pathways.
\subsection{Network architecture}
\subsection{Transfer learning}
\subsection{Cross-validation and hyperparameter tuning}
Due to the very small available datasets we found the performance of a given model can vary significantly based on the test splitting of the data. In addition, the performance can vary based on the random seeds used in the stochastic minimization of the loss function. We also found that the performance can depend significantly on model hyper-parameters. If these are tuned to optimize the performance on a particular test splitting of the data this will result in an overly optimistic estimate of the performance when trained on unseen data.
\\ \indent To account for this we have developed a cross-validation procedure. This is used both to tune hyper-parameters and to make an unbiased estimate the model performance with a faithful uncertainty estimate. This procedure allows a fair comparison of different models.
\\ \indent We employ a procedure of nested-random-cross-validation \cite{}. A set of randomly chosen test datasets  are selected. In each case the remaining data is used to tune the model hyper-parameters. For this we use a grid search with random-cross validation. For every hyper-parameter set the data is is partitioned into a validation and a training set. The model training is run using the validation set to decide the early stopping of the training and to estimate the performance. This is then repeated a number of times to obtain an estimate of the mean performance and uncertainty for this hyper-parameter set. The hyper-parameter set with the best model performance are then used to train the model, with the performance judged on the test dataset. The distribution of this testing performance give an unbiased estimate of the model performance, with a reliable uncertainty estimate. This procedure is very computationally intensive and has been implemented using batch job submission in parallel on the Myriad cluster at UCL.
\\ \indent There are a number of hyper-parameters potentially affecting the model performance. For example the learning rate of the gradient descent and learning rate schedule, L2 regularization parameters, drop-out parameters etc. These hyper-parameters can broadly be divided into two classes, those which control the minimization of the loss function and those which add some terms / modify it to prevent over-fitting. We found most of the hyper-parameters are correlated so focused on two for a optimization, the learning rate and the regularization parameter which are tuned using a grid search.
%\input{method}

%---------------------------------------------------------------------


\section{Results}
\label{sec:results}
{\color{red}Toby: Here we could split up each of the 3 sub-projects but i think it would be more cohesive to just describe the results in parallel with the two network archtiutres}
\\
\subsection{Comparison of data types and architecture}
We train the model independently for three different data types (see \ref{sec:data}), gene-expression, copy number violation and mutations. This is performed for both the Reactome and Gene Ontology network architure.
\\
{   \color{red}
Plots to include:
\begin{itemize}
    \item Example scatter plots ?
\item Performance for each of the 6 models?
\item Hyperparameter scan plot?
\end{itemize}}
\subsection{Transfer learning}
%\input{results}
%---------------------------------------------------------------------

\section{Conclusion}
\label{sec:conclusion}


%\input{conclusion}
%---------------------------------------------------------------------

\end{document}
